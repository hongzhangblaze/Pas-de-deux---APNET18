\abstract{
Despite continued efforts toward building high bandwidth, low cost datacenter networks with reconfigurable optical fabrics, the impact of optical networks on datacenter applications has received little attention.
Given the constraints of optical networks and the semantics of datacenter applications, we believe the network-application intersection to be the next innovation hotspot. 
In this paper, we specifically focus on data-parallel applications for two primary reasons: they are a natural fit to exploit high bandwidth optical fabrics, and they often form structured communication patterns or coflows.

We show that configuring circuits in reaction to changing traffic patterns is not enough. 
Efficient scheduling of even a single coflow in optical networks should be a \emph{``Pas de deux''}\footnote{A pas de deux is a dance duet in which two dancers perform ballet steps together.} -- \emph{ a joint shaping of not only the underlying circuit, but also the application's traffic demand.}
Our preliminary evaluation with a production trace shows that joint shaping is on average within 1.18$\times$ of the optimal and performs $30\%$ better than solutions that configure circuits in application-agnostic fashions.
We further extend our analysis to inter-coflow scheduling and propose a layered solution that jointly considers circuit reconfiguration, coflow prioritization, as well as flow rate and route assignments.
}
