\section{Introduction}
\label{sec:intro}
In recent years, a growing body of datacenter architecture designs \cite{mordia-ptl,projector,megaswitch,osa,cthrough} have proposed interconnecting Top-of-Rack (ToR) switches with optical fabrics.
Although these designs provide high bisection bandwidth, low power consumption, and reduced cabling complexity, they suffer from high circuit reconfiguration delay of up to tens of milliseconds \cite{megaswitch}.
Yet, it is still unclear how they affect the end-to-end performance of datacenter applications.

Scheduling applications on the state-of-the-art optical architecture \cite{projector,megaswitch} differs from previous circuit scheduling problems \cite{eclipse,sunflow,solstice} in the following three aspects:
\begin{icompact}
\item{\textbf{Application semantics}: Most existing circuit scheduling algorithms aim to maximize the throughput of an aggregated traffic matrix. %consists of multiple flows.
    However, this may not directly benefit application-level communication
performance \cite{varys} because each application can have its own traffic matrix and its own performance objective.
The coflow abstraction \cite{coflow} captures such semantics of communication between two groups of machines in successive computation stages.
}
\item{\textbf{Traffic pattern}: Most existing circuit scheduling designs assume that the traffic demand matrix is \emph{sparse} \cite{cthrough,eclipse}. However, many coflows involve wide-spread communication among thousands of machines \cite{varys}.
    }
\item{\textbf{Connectivity:} Most previous work adopt a network model, where one ToR can only establish one exclusive circuit connection to another ToR.
However, to support wide-spread communication patterns, recent optical datacenter architecture designs \cite{projector,megaswitch} allow increased connectivity among ToRs.
For example, one ToR can simultaneously establish tens (ProjecToR \cite{projector}) to hundreds (MegaSwitch \cite{megaswitch}) of circuit connections to other ToRs.}
\end{icompact}
Overall, the presence of coflows in data-parallel applications and the advances in optical designs greatly complicate the circuit scheduling problem.
Minimizing the average coflow completion time (CCT) requires coordinated scheduling of flows within each coflow as well as prioritized scheduling among multiple coflows.
The rich connectivity among ToRs further increases the solution space.

In this paper, we start by investigating an important special case that involves serving a single coflow on optical fabrics.
We observe that although the rich connectivity brings new opportunity to serve coflows without reconfiguration,
obtaining a circuit configuration that well matches a coflow's traffic pattern is not always possible.
Such mismatch between the coflow and circuit structure greatly increases the CCT ($\S$\ref{sec:alg1}).

To alleviate this mismatch, we instead ask the following question:
\emph{can we reshape the coflow traffic demand matrix to match the circuit configuration?}
In particular, we notice that the demand matrix can be effectively reshaped by rerouting traffic via other ToRs as relays.
Multi-hop rerouting is not new for optical networks.
However, unlike previous work \cite{osa} that employ multi-hop routing to provide overall connectivity, we leverage a fine-grained, coflow-specific multi-hop routing scheme to redistribute the original coflow traffic demand among ToRs ($\S$\ref{sec:alg2}).
However, while reshaping coflows can help in mitigating the mismatch and improving CCT, its effectiveness depends on underlying circuit configurations.

Given that neither coflow nor circuit (re)shaping alone efficiently minimizes CCT, we reformulate the problem as \emph{a joint shaping of both circuit configuration and coflow traffic demand}.
We show through examples that such joint scheduling brings extra flexility to find a good matching, and effectively minimizes the CCT ($\S$\ref{sec:alg3}).
Moreover, we design a heuristic that effectively calculates the joint scheduling by
iteratively:
(i) releasing under-utilized circuits by reshaping the coflow demand;
(ii) configuring the released circuits to serve the bottleneck source-destination ToR pair.
Preliminary evaluation with a production trace shows that
joint shaping is on average within 1.18$\times$ of the optimal and performs $30\%$ better in comparison to algorithms that only configure circuits ($\S$\ref{sec:eval}).

We conclude by extending our analysis to the general case of inter-coflow scheduling ($\S$\ref{sec:alg4}).
Specifically, we find each of the following factors: circuit configuration, routing, rate allocation, and coflow prioritization to play important roles, leading to a joint scheduling problem with exponential complexity.
To perform efficient scheduling with reasonable complexity, we design a framework that effectively decouples the scheduling into three successive steps -- circuit configuration, coflow permutation, and rate/routing.
Each step is further transformed into the single coflow scheduling problem we have already formulated.
We leave the in-depth evaluation under the general case as future work.
